\section{Zusammenfassung}

In diesem Experiment wurden erste Erfragungen mit RFS gesammelt, dabei konnte einige versuche mit dem Messinstrument und der zugehörigen Software der Firma Fischer durchgeführt werden. Die leichte Handhabung und Integration dieses Instrumentes haben die Durchführung des Experiments stark vereinfacht, jedoch haben sich die Limitierung der Software in der anschließenden Auswertung bemerkbar gemacht. Da die produzierten Spektral nicht in einem Format exportiert werden können mit der sie im Nachhinein durch eine andere Software analysiert werden können. Dies schließt eine neue Analyse des Spektrums aus und erfordert eine neue Vermessung der Probe falls wichtige Aspekte bei der ursprünglichen Analyse nicht berücksichtigt wurden. \\
Bei dem Experiment ist es gelungen eine Stahlprobe auf ihre Komposition zu analysieren, es war jedoch nicht möglich alle Bestandteile zu identifizieren. Dies ist vor allem auf den Mangel an Erfragung mit RFS Spektroskopie zurückzuführen. Es ist davon auszugehen das mit mehr Übung es möglich ist mit diesem Messinstrument die meisten elementaren Bestandteile in einer Probe aus Stahl zu identifizieren.  \\
Bei der Untersucheng von einen C-Ni Multilayer Spiegel war es möglich die Schichtdicken zu bestimmen und auch die Gleichheit dieser Schichtdicken zu validieren. Mit dieser Information war es möglich Reflektionscharakteristiken des Spiegels zu berechnen. \\
Es wurde zudem versucht Verunreinigungen in Trinkwasser der Technischen Universität Berlin nachzuweisen, diese ist zum Glück nicht gelungen, es konnten lediglich Elemente nachgewiesen werden die Normalerweise im Trinkwasser vorhanden sind. Eine qualitative Analyse des Trinkwassers ist jedoch nicht gelungen da eine starke Verunreinigung des Probenhalters festgestellt wurde. Der Ursprung dieser Verunreinigung auf dem Behälter kann mit den aktuellen Informationslagen nicht abschließend geklärt werden. Hierfür sind weiter Nachforschungen nötig um Gewissheit zu schaffen. Es wird dennoch davon ausgegangen das der Behälter mit Metall Pulver verunreinigt wurde.\\
Dieser experimentale Aufbau hat einen kleinen Einblick in die Möglichkeiten der RFA gegeben. Es ist ein Verfahren das vielseitig eingesetzt werden kann und wird. Die weitere Beschäftigung an diesem Verfahren ist von großem Interesse.
