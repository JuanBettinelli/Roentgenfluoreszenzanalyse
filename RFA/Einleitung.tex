\section{Einleitung}
Röntgenfluoreszenzanalyse (RFA) ist eine sehr verbreitete Materialanalyste Methode, die eine Probe qualitativ und quantitativ auf ihre elementare Zusammensetzung untersuchen kann. Dabei wird sich die unterscheidbaren Röntgenfluoreszenzeigenschaften der Elemente zunutze gemacht. Eine Einführung in die Funktionsweise dieser Methode wird in diesem Bericht erläutert, zusammen mit experimentellen Beispielen der Anwendungsmöglichkeiten dieser Methode. Dafür wurde eine Reihe von Versuchen durchgeführt.\\
Ein Streuexperiment auf einer ABS Probe wurde verwendet, um die Eigenschaften des Anregungsstrahls des Instrumentes zu identifizieren und zu ermitteln wie diese mit Hilfe von Filtern und Veränderungen der Anregungsenergien modifiziert werden kann.
Die Materialanalyse Eigenschaften der RFA Methode wurden anhand einer Stahlprobe und eines C-Ni Multilayer Spiegel getestet. Die Ergebnisse dieser Versuche werden hier thematisiert. Anhand einer Wasserprobe des Trinkwassers der Technischen Universität Berlin wurde ein großer Vorteil des Verfahren ausgenutzt. Dies ist die Möglichkeit Flüssigkeiten zu analysieren. Dabei wurden einige Spurenelemente im Trinkwasser gefunden, welches hier diskutiert werden.

